\documentclass{article}

\long\def\testeq#1#2{\def\tmp{#2}\ifx#1\tmp\else \toks0={#1wantd: ->#2.}
\typeout{\the\toks0}\show #1\uerror\fi}

\begin{filecontents+}{kvp.plt}
\RequirePackage{xkeyval}
\newif\iflandscape
\DeclareOptionX{landscape}{\landscapetrue}
\def\upo{}
\DeclareOptionX*{\xdef\upo{\upo,\CurrentOption}}

\ProcessOptionsX \relax
\end{filecontents+}

\begin{filecontents+}{kvo.plt}
  \ProvidesPackage{kvo}[2007/10/18 kvoptions test]
  \RequirePackage{ifpdf} \RequirePackage{kvoptions}
  \SetupKeyvalOptions{ family=MCS, prefix=MCS@ }
  \DeclareBoolOption[true]{BA} \DeclareBoolOption[false]{BB}
  \DeclareBoolOption[true]{BC} \DeclareBoolOption{BD}
  \DeclareBoolOption[true]{BE} \DeclareBoolOption[true]{BF}
  \DeclareBoolOption[true]{BG} \DeclareBoolOption[true]{BH}
  \DeclareComplementaryOption{BNH}{BH} \DeclareBoolOption[true]{BI}
  \DeclareComplementaryOption{BNI}{BI}
  \DeclareVoidOption{VA}{\let\foo\relax} \DeclareVoidOption{VB}{}
  \DeclareStringOption{SA} \DeclareStringOption[foo]{SB}
  \DeclareStringOption[foo]{SC}[bar]
  \DeclareStringOption[foo]{SD}[bar] \let\foo\undefined
  \def\@@relax{\relax} \def\theopts{} \DeclareDefaultOption{%
    \edef\theopts{\theopts,\CurrentOptionKey+%
      \ifx\CurrentOptionValue\@@relax
      RELAX\else\CurrentOptionValue\fi+\CurrentOption}}
  \ProcessKeyvalOptions* \edef\xxa{\CurrentOption}
  \edef\xxb{\CurrentOptionKey} \edef\xxc{\CurrentOptionValue}
\end{filecontents+}

\usepackage{epigraph,esdiff,calc}
\usepackage{xkeyval,textcomp}
\usepackage{graphicx}
\usepackage[dvipsnames]{color}
\usepackage{fancybox}
\usepackage{fixme}
\usepackage{fink}
\usepackage{fixfoot}
\usepackage[landscape,unknownA,unknownB=c]{kvp}
\usepackage[BE =false,BC=false,BF=true,BG=false,BNH=true,BNI=false,VA,VB,
   SC=yfoo,SD,ua,ub=c]{kvo}
\foo
\DeclareFixedFootnote{\blah}{blah \didi blah}
\DeclareFixedFootnote*{\blih}{blih da blih}
\begin{document}

\unless\iflandscape\bad\fi
\edef\xfoo{\detokenize{,unknownA,unknownB=c}}
\edef\yfoo{\detokenize\expandafter{\upo}}
\ifx\yfoo\xfoo\else\bad\fi
% BA is true BB false BC false BD false
\makeatletter
%%%\testerUnknownOptions dans xkeyval
\edef\xfoo{\detokenize{,ua+RELAX+ua,ub+c+ub=c}}
\edef\yfoo{\detokenize\expandafter{\theopts}}
\ifx\yfoo\xfoo\else\bad\fi

\let\res\bad
\ifMCS@BA \unless\ifMCS@BB \unless\ifMCS@BC \unless\ifMCS@BD
\unless\ifMCS@BE  \ifMCS@BF\unless\ifMCS@BG
\let\res\relax \fi\fi\fi\fi\fi\fi\fi \res
\let\res\bad
\unless\ifMCS@BH\ifMCS@BI\let\res\relax \fi\fi\res
\def\xfoo{}\ifx\MCS@SA\xfoo\else\bad\show\MSC@SA\fi
\def\xfoo{foo}\ifx\MCS@SB\xfoo\else\bad\fi
\def\xfoo{bar}\ifx\MCS@SD\xfoo\else\bad\fi
\def\xfoo{yfoo}\ifx\MCS@SC\xfoo\else\bad\fi

\let\xfoo\KV@MCS@SC\let\yfoo\KV@MCS@SC@default
{
\DisableKeyvalOption[local]{MCS}{SC}
\ifx\KV@MCS@SC\relax\else\bad\fi
\ifx\KV@MCS@SC@default\relax\else\bad\fi
}
\ifx\KV@MCS@SC\xfoo\else\bad\fi
\ifx\KV@MCS@SC@default\yfoo\else\bad\fi
{\DisableKeyvalOption[global]{MCS}{SC}}
\ifx\KV@MCS@SC\relax\else\bad\fi
\ifx\KV@MCS@SC@default\relax\else\bad\fi
\let\KV@MCS@SC\xfoo\let\KV@MCS@SC@default\yfoo
{\DisableKeyvalOption{MCS}{SC}}
\ifx\KV@MCS@SC\relax\else\bad\fi
\ifx\KV@MCS@SC@default\relax\else\bad\fi
\let\KV@MCS@SC\xfoo\let\KV@MCS@SC@default\yfoo
% Testing ignore
{
\DisableKeyvalOption[local,action=ignore]{MCS}{SC}
\ifx\KV@MCS@SC\@gobble\else\bad\fi
\ifx\KV@MCS@SC@default\@empty\else\bad\fi
}
\ifx\KV@MCS@SC\xfoo\else\bad\fi
\ifx\KV@MCS@SC@default\yfoo\else\bad\fi
{\DisableKeyvalOption[action=ignore]{MCS}{SC}}
\ifx\KV@MCS@SC\@gobble\else\bad\fi
\ifx\KV@MCS@SC@default\@empty\else\bad\fi
{
  \DisableKeyvalOption[class=foo,global=false,action=error]{MCS}{SC}
  \def\xfoo#1{\KVO@disable@error{foo}\KVOdyn@ext {SC}}
  \def\yfoo{\KV@MCS@SC{}}
  \ifx\xfoo\KV@MCS@SC\else \bad\fi
  \ifx\yfoo\KV@MCS@SC@default\else \bad\fi
  \DisableKeyvalOption[package=bar,local,action=warning]{MCS}{SC}
  \def\xfoo#1{\KVO@disable@warning{bar}\KVOdyn@ext {SC}}
  \ifx\xfoo\KV@MCS@SC\else \bad\fi
  \ifx\yfoo\KV@MCS@SC@default\else \bad\fi
}
\def\yfoo{\KV@MCS@SC{}}
\def\xfoo#1{\KVO@disable@warning{bar}\KVOdyn@ext {SC}}
\ifx\xfoo\KV@MCS@SC \bad\fi
\ifx\yfoo\KV@MCS@SC@default\bad\fi
\DisableKeyvalOption[package=bar,local,action=warning]{MCS}{SC}
\KV@MCS@SC{foo}
\makeatother

\epigraphhead[70]{\epigraph{Star crossed lovers.}{\textit{The Bard}}}
\epigraph{Example is the school of mankind, and they will learn at no other}
{\textit{Letters on a Regicide Peace}\\ \textsc{Edmond Burke}}
\begin{epigraphs}
\qitem{A text}{An author}
\qitem{A second text}{An author}
\end{epigraphs}

\newenvironment{Foo}{\begin{xmlelement}{Foo}A}{B\end{xmlelement}}
\newenvironment{Foo+ Bar_}{\begin{xmlelement}{special}A}{B\end{xmlelement}}
\newenvironment{*$�$}{\begin{xmlelement}{VerySpecial}A}{B\end{xmlelement}}
\def\foo{Foo}\begin{\foo}X\end{\foo}
\def\foo{Foo+ Bar_}\begin{\foo}X\end{\foo}
\def\foo{*$�$}\begin{\foo}X\end{\foo}


\def\T{\diff{f}{x}}$\T_{\T^\T}$
\def\T{\diff[2]{f}{x}}$\T_{\T^\T}$
\def\T{\diff*{f}{x}{x_0}}$\T_{\T^\T}$
\def\T{\diff*[2]{g}{y}{0}}$\T_{\T^\T}$

$\diffp{f}{x}=\diffp[2]{f}{x}= \diffp*{p}{V}{T}= \diffp{f}{{x}{y^2}}=
\diffp{f}{{x^2}}=\diffp*{f}{{x^2}{y^3}}{z}$ 

\[\diffp{f}{x} =\diffp[2]{f}{x}=\diffp*{p}{V}{T}=\diffp{f}{{x}{y^2}}=
\diffp{f}{{x^2}}=\diffp*{f}{{x^2}{y^3}}{z}\]



\def\myscale{.3}
\includegraphics[height=15cm,width=\columnwidth]{A}
\includegraphics[height=15cm,totalheight=\columnwidth]{A}
\includegraphics[height=20cm,width=\textheight]{AA}
\includegraphics[height=18cm,width=\textheight-2cm]{AA}
\includegraphics[height=1.5cm,width=.1\linewidth]{B}
\includegraphics[height=7.5cm,width=.5\textwidth]{C}
\includegraphics[height=7.5cm,width=.5\textwidth]{C}
\includegraphics[bb=10 20 30  40,clip=true,keepaspectratio=true,draft=true,hiresbb=true,origin=c]{D}
\includegraphics[bb=10pt 20pt 30pt  40pt,clip=true,keepaspectratio=true,draft=true]{D}
\includegraphics[bbllx= 9.9626,bblly= 20pt,bburx= 30,bbury=40,clip=false,keepaspectratio=false,draft=false,type=aa,ext=bb,read=cc,command=dd,hiresbb=false]{E}
\includegraphics[natwidth= 3,natheight=40,clip=,hiresbb,keepaspectratio=,scale=\myscale,draft=]{F}

{\setkeys{Gin}{width=20pt}\includegraphics{G}}
\graphicspath{foo/bar}
\DeclareGraphicsExtensions{ext1,ext2,ext3}
\DeclareGraphicsRule{ext}{type}{read-file}{command}

\makeatletter
\fontdimen1\font=.33pt
\dimen@=10pt
$\kern\strip@pt\fontdimen1\font\dimen@$
\edef\foo{\strip@pt\dimen@}
\def\xfoo{10}\ifx\foo\xfoo\else\show\foo\bad\fi
\edef\foo{\strip@pt\fontdimen1\font}
\def\xfoo{0.33}\ifx\foo\xfoo\else\show\foo\bad\fi
\rotatebox{30}{Text}
\rotatebox[x=1pt,y=2pt,origin=c]{30}{Text}
\rotatebox{30}{\includegraphics{x}}
\rotatebox[x=1pt,y=2pt,origin=c,units= -360]{30}{\includegraphics{x}}
\reflectbox{foo}
\resizebox{1in}{2in}{Some text}
\resizebox{1in}{!}{Some text}
\resizebox{1in}{\height}{Some text}

%% Testing colors
{\color{red} textA \par textB {\it testC} textD}out
{\color{Red} textD \color {Blue} textE \color[rgb]{.1,.2,.3} textF}out
{\color[named]{Yellow} textG {\normalcolor this} and that}
\pagecolor{green}\pagecolor[rgb]{0,1,0}
\definecolor{myred}{named}{WildStrawberry}
\definecolor{mygreen}{rgb}{.1,.2,.3}
\colorbox{red}{textH etc\textcolor{myred}{redtext}etc\textcolor{mygreen}{greentext}}
\colorbox[cmyk]{0,1,1,1}{textI}
\fcolorbox{red}{green}{textJ}
\fcolorbox[cmyk]{0,1,1,1}{.1,.2,.3,.4}{textK}

$x\thinspace y\,z$\thinspace a\,b

%% textcomp
\leavevmode \textcompwordmark\textcopyleft \par


\makeatletter
\gdef \newlabel#1#2{\newlabelxx{#1}#2}
\gdef \newlabelxx#1#2#3#4#5#6{\@namedef{label@value@#1}{#2}\@namedef{label@title@#1}{#4}}
\def\myref#1{\xbox{ref}{\XMLref{#1}\@nameuse{label@title@#1}%
\XMLaddatt{value}{\@nameuse{label@value@#1}}}}

\newlabel{labA}{{1_1}{1}{Introduction\relax }{subsection.1.1}{}}
\newlabel{labB}{{1.2}{1}{Continuation\relax }{subsection.1.2}{}}

Text. \anchor\label{labA}
\myref{labA}\myref{labB}
\anchor\label{labB}

\def\foo{\@reevaluate\ca\cb}
\def\ca#1{A#1}\def\cb#1#2{B#1C#2=}
\foo 123
\foo {1}23\foo {12}23


\expandafter\def\expandafter\getuserid \detokenize{cite}:#1\\{#1}

\def\expanduserid#1{\expandafter\getuserid \detokenize{#1}\\}
\renewenvironment{citation}[1]
{\begin{xmlelement}{bib\textunderscore entry}#1\NCcitation}
{\end{xmlelement}}
\def\NCcitation#1#2#3#4{%
\XMLaddatt{doctype}{#4}%
\XMLaddatt{id}{#2}%
\XMLaddatt{user-id}{\expanduserid{#1}}%
\@nameuse{bibref#1}}

\@namedef{bibrefcite:ab}{FOO}
\begin{citation}{text}{cite:foo^bar}{A2}{A3}{A4}ETC\end{citation}
\begin{citation}{text}{cite:ab}{A2}{A3}{A4}ETC\end{citation}
    
XYX \^M(\^^M(
            blabla.\
            (taratata)
           blabla.\
            (taratata)
$\overline{\bmod}$

%%%%%%%%%%%%%%%%%%%%%%%%%%%%%%%%%%%%%%%%%%%%%%%%%%
Testing fancybox

\shadowbox{\Large\bf New Glarus Birdwatch}
\doublebox{\Large\bf New Glarus Birdwatch}
\ovalbox{\Large\bf New Glarus Birdwatch}
\Ovalbox{\Large\bf New Glarus Birdwatch}
x\cornersize{.7}\cornersize*{17cm}x
\begin{Sbox}\begin{minipage}{200Pt}blah\end{minipage}\end{Sbox}
\fbox{\TheSbox}
\newenvironment{fminipage}{\begin{Sbox}\begin{minipage}}
{\end{minipage}\end{Sbox}\fbox{\TheSbox}}
\begin{fminipage}{150pt} Since the former doesn't use braces to delimit
the context of the box, $\ldots$
\end{fminipage}

\newlength\mylength
\begin{equation}\fbox{$\displaystyle \int $}\end{equation}

\newenvironment{FramedEqn}
 {\setlength{\fboxsep}{15pt}
   \setlength{\mylength}{\linewidth}
   \addtolength{\mylength}{-2\fboxsep}
   \addtolength{\mylength}{-2\fboxrule}
   \Sbox
   \minipage{\mylength}
      \setlength{\abovedisplayskip}{0pt}
      \setlength{\belowdisplayskip}{0pt}
      \equation}%
    {\endequation\endminipage\endSbox
      \[\fbox{\TheSbox}\]}


\newenvironment{FramedTable}
{\begin{table}[h]
    \begin{Sbox}
      \setlength{\mylength}{\textwidth}
      \addtolength{\mylength}{-2\fboxsep}
      \addtolength{\mylength}{-2\fboxrule}
      \begin{minipage}{\mylength}}
      {\end{minipage}\end{Sbox}\fbox{\TheSbox}\end{table}}
x\fbox{$y$}z

\fbox{\[ x\]}


\fbox{%
  \begin{Beqnarray*}
    x &=&y\\
    y& >& x\\
    \int_4^5 f(x)dx &=&\sum_{i\in F} x_i
  \end{Beqnarray*}
}
\fbox{%
  \begin{Beqnarray}
    x &=&y\\
    y& >& x\\
    \int_4^5 f(x)dx &=&\sum_{i\in F} x_i
  \end{Beqnarray}
}
\begin{verse} etc \end{verse}

\begin{table}[h]
  \begin{center}
    \fbox{%
      \begin{minipage}{.8\textwidth}
        \begin{center}
          \begin{tabular}{rl}
            foo & bar
          \end{tabular}
        \end{center}
        \caption{A table of foo and bar}
      \end{minipage}}
  \end{center}
\end{table}


\begin{FramedTable}
  \begin{center}
    \begin{tabular}{rl}
      foo & bar
    \end{tabular}
  \end{center}
  \caption{A table of foo and bar}
\end{FramedTable}


\fbox{\begin{tabular}{c} x & y  \end{tabular}}

\setlength{\fboxsep}{10pt}
\fbox{
\begin{Bcenter}
Love of life\\
and other short stories\\[2pt]
by Policarpa Salabarrieta 
\end{Bcenter}}

My list: \fbox{
\begin{Bflushleft}[b]
Galanga root\\ Coconot\\Tempeh
\end{Bflushleft}}

\begin{landfloat}{table}{\rotateleft}
Hello, world
\end{landfloat}

\fixme{A}\fxnote{B}\fxwarning{C}\fxerror{D}
\begin{anfxnote}[A]aa\end{anfxnote}
\begin{anfxwarning}[B]bb\end{anfxwarning}
\begin{anfxerror}[C]cc\end{anfxerror}
\begin{afixme}[D]dd\end{afixme}

\def\didi{di }
    \noindent       here we are again\blah
\par\noindent       happy as can be\blah
\clearpage\noindent all good friends\blah OK
\par\noindent       and jolly good company\blah\blih OK
\footnote{xyz}

\end{document}

%%%%%%%%%%%%%%%%%%%%%%%%%%%%%%%%%%%%%%%%%%%%%%%%%%%%%%%%%%%%%%%%%%%%%%%%%%%%
\begin{landfloat}{table}{\rotateleft}
\begin{equation}
  x+y=z
\end{equation}
\end{landfloat}




This does not yet work
You cannot put display math in a hbox


\[
  \setlength{\fboxsep}{15pt}
  \setlength{\mylength}{\linewidth}
  \addtolength{\mylength}{-2\fboxsep}
  \addtolength{\mylength}{-2\fboxrule}
  \fbox{\parbox{\mylength}{
      \setlength{\abovedisplayskip}{0pt}
      \setlength{\belowdisplayskip}{0pt}
      \begin{equation} x + y = z \end{equation}}}\]



\begin{FramedEqn}
  \Rightarrow P\sim\xi(P_\gamma)-\frac{1}{3}
\end{FramedEqn}
\fi


  
\iffalse

\fi



\end{document}


